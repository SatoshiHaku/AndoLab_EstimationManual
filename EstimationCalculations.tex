\documentclass[12pt]{jsbook}
%\documentclass{jaeticle}
\bibliographystyle{unsrt}
\usepackage{amsmath,amssymb}


\begin{document}
\chapter{ST-FMR測定における各物理量の計算式のまとめ}
スピントルク強磁性共鳴法(ST-FMR)を用いたスピン軌道トルク(SOT)生成効率の定量方法を簡単に述べていく。解析方法の種類は主に三つに分けられる。ここでは、ST-FMR信号が以下のようなローレンツ関数で表現されること既知とする。
\begin{align}
	V_{mix}=V_{asy}\frac{W(\mu_0H-\mu_0H_{FMR})}{(\mu_0H-\mu_0H_{FMR})^2+W^2}+V_{sym}\frac{W^2}{(\mu_0H-\mu_0H_{FMR})^2+W^2}
\end{align}
一つ目はFLトルクが無視できるときは、
\begin{align}
	\theta_{\rm{SH}}^{\rm{eff}}=\xi_{\rm{DL}}=\frac{V_{sym}}{V_{asy}}\frac{e\mu_0M_st_{FM}t_{NM}}{\hbar}\sqrt{1+\frac{M_{eff}}{H_{FMR}}}
\end{align}
と、DLトルク効率を定量することができる。しかし、FLが無視できない時は、強磁性体膜厚が異なる複数サンプルもしくは単一サンプルから定量する方法がある。まず、単一サンプルから求める方法を説明する。
\section{A single sample}
単一サンプルからDLトルク効率を求める方法は主に二つあげられる。一つ目はST-FMR信号の$S$成分から直接計算する方法がある。これは、試料に流れている電流$I_{RF}$,試料のAMR変化量$\Delta R$, 飽和磁化$M_s$が別途見積もる必要がある。もう一つは、緩和変調(DM)を利用した定量方法がある。
\subsection{current calibration method}
試料を流れる電流はジュールヒーティングを利用した方法と、ベクトルネットワークアナライザを利用して、試料から反射するマイクロ波電流から算出する方法が挙げられる。これらを用いることで資料に流れるマイクロ波電流$I_{RF}$を見積もれる。\\
それでは、どのように$S$成分からDLトルク効率を算出する方法を述べていく。
damping-like torqueとfield-like torqueの有効磁場を$H_{L} \propto \xi_{\mathrm{DL}} J_{e}(\hat{\sigma} \times \hat{m})$と$H_{T} \propto\xi_{\mathrm{FL}} J_{e} \hat{\sigma}$のように定義すると
\begin{eqnarray}
\xi_{\mathrm{DL}(\mathrm{FL})}(T)=\left(\frac{2 e}{\hbar}\right) 4 \pi M_{s}(T) t_{\mathrm{FM}}^{\mathrm{eff}}\left(\frac{H_{L(T)}}{J_{e}}\right)
\end{eqnarray}
これらの有効磁場は,面内に磁場を印加して,電流を$x$軸に流しているとすると~\cite{PhysRevB.92.214406}
\begin{eqnarray}
V_{\mathrm{sym}}=\frac{I \Delta R}{2} A_{\mathrm{sym}} H_\mathrm{L}\cos\theta \sin 2 \theta \\
V_{\mathrm{asy}}=\frac{I \Delta R}{2} A_{\mathrm{asy}}\left(H_{y} \cos \theta-H_{x} \sin \theta\right) \sin 2 \theta
\end{eqnarray}
普通は$H_x$(電流に並行な方向にスピン偏極した角運動量が磁化に受け渡されるってこと)はないので,$H_y = H_\mathrm{T}$となる.このとき
\begin{eqnarray}
A_{\mathrm{sym}}=\frac{\gamma\left(H_{\mathrm{res}}+H_{1}\right)\left(H_{\mathrm{res}}+H_{2}\right)}{\omega \Delta H\left(2 H_{\mathrm{res}}+H_{1}+H_{2}\right)} \\
A_{\mathrm{asy}}=\frac{\left(H_{\mathrm{res}}+H_{1}\right)}{\mu_{0} \Delta H\left(2 H_{\mathrm{res}}+H_{1}+H_{2}\right)}
\end{eqnarray}
である.
磁気的な自由エネルギー$F$を
\begin{eqnarray}
F[\theta, \phi]=&\nonumber F_{\mathrm{Zeeman}}[\theta, \phi]+F_{\mathrm{surf}}[\theta, \phi]+F_{\mathrm{shape}}[\theta, \phi] \\
&+F_{U}[\theta, \phi]+F_{\mathrm{exch}}[\theta, \phi] \\\nonumber
=& \mu_{0}\left(H+H_{\mathrm{rot}}\right) M d_{F}\left(\sin \phi \sin \phi_{H} \cos \left(\theta-\theta_{H}\right)\right.\\\nonumber
&\left.+\cos \phi \cos \phi_{H}\right)+\left(\mu_{0} M^{2} d_{F} / 2-K_{S}\right) \cos ^{2} \phi \\\nonumber
&-K_{U} d_{F} \sin ^{2} \phi \cos ^{2}\left(\theta-\theta_{\mathrm{uni}}\right) \\
&-\mu_{0} M d_{F} H_{\mathrm{ex}} \cos \left(\theta-\theta_{\mathrm{exch}}\right) \sin \phi
\end{eqnarray}
としたとき,共鳴条件
\begin{eqnarray}
\left(\frac{\omega}{\gamma}\right)^{2}=\frac{1}{M^{2} d_{F}^{2} \sin ^{2} \theta}\left[\left(\frac{\partial^{2} F}{\partial \theta^{2}}\right)\left(\frac{\partial^{2} F}{\partial \phi^{2}}\right)-\left(\frac{\partial^{2} F}{\partial \phi \partial \theta}\right)^{2}\right]
\end{eqnarray}
を考えると,
\begin{eqnarray}
\left(\frac{\omega}{\gamma}\right)^{2}=\mu_{0}^{2}\left(H+H_{1}\right)\left(H+H_{2}\right)
\end{eqnarray}
となる.ただし
\begin{eqnarray}
H_{1}&=&H_{\mathrm{rot}}+M_{\mathrm{eff}}+H_{\mathrm{exch}} \cos \left(\theta-\theta_{\mathrm{exch}}\right)+H_{U} \cos ^{2}\left(\theta-\theta_{U}\right) \\
H_{2}&=&H_{\mathrm{rot}}+H_{\mathrm{exch}} \cos \left(\theta-\theta_{\mathrm{exch}}\right)+H_{U} \cos \left[2\left(\theta-\theta_{U}\right)\right]
\end{eqnarray}
である.さらに,$H_1$が$H_\mathrm{res}$に対して小さい場合,$H_\mathrm{res}+H_1 \approx M_\mathrm{eff}$とできて,共鳴条件は
\begin{eqnarray}
\mu_{0} H_{\mathrm{res}}=&\nonumber \left(\frac{\omega}{\gamma}\right)^{2} \frac{1}{\mu_{0} M_{\mathrm{eff}}}-\mu_{0} H_{\mathrm{rot}}-\mu_{0} H_{\mathrm{exch}} \cos \left(\theta-\theta_{\mathrm{exch}}\right) \\
&-\mu_{0} H_{U} \cos \left[2\left(\theta-\theta_{U}\right)\right]
\end{eqnarray}
と書ける.ただ,通常反強磁性体を使用していない限りexchange bias $H_{\mathrm{exch}}$はなく(元論文は反強磁性体IrMnを使用しているため~\cite{PhysRevB.92.214406}),一軸異方性$H_{U}$もなく,回転磁気異方性$H_\mathrm{rot}$(元論文では多結晶IrMnのグレインがPyと結合することによって生じている~\cite{PhysRevB.92.214406})もないので,結局Py/Ptなどの試料では$H_1=M_\mathrm{eff}$で,$H_2=0$となる.


結局,$A_{\mathrm{sym}}$と$A_{\mathrm{asy}}$は
\begin{eqnarray}
A_{\mathrm{sym}}=\frac{\gamma\left(H_{\mathrm{res}}+M_\mathrm{eff}\right)H_{\mathrm{res}}}{\omega \Delta H\left(2 H_{\mathrm{res}}+M_\mathrm{eff}\right)} \\
A_{\mathrm{asy}}=\frac{\left(H_{\mathrm{res}}+M_\mathrm{eff}\right)}{\mu_{0} \Delta H\left(2 H_{\mathrm{res}}+M_\mathrm{eff}\right)}
\end{eqnarray}
と簡単になり,これを使えばいい. これから、$V_{sym}$から$\Delta R, I, M_s$がわかれば、$H_{DL}$が逆算することができるため、$\xi_{DL}$が算出できる。
\subsection{緩和変調(DM)}
緩和変調法とは、ST-FMRを測定している際に、外部からDC電流を流しながら測定する際に、SOTによって、電流の極性によってトルクの向きが反転するために、ST-FMRスペクトルの線幅が変化する。この線幅の変化率から$\xi_{DL}$を見積流ことができる。試料に流れる電流は外部から制御できるが、SOT生成源に流れる電流量を見積もるため、$\rho_{NM(FM)}$が必要となる。線幅の変化率と$\xi_{DL}$は次のような関係がある。また$M_s$も見積もる必要がある\footnote{ $\displaystyle M_{eff}=M_s-\frac{2K_s}{\mu_0M_st_{FM}}$の関係があり、$K_s\ll 1$であれば、$M_{eff}\approx M_s$と近似すれば、$M_s$を別途求める必要はない。}。
\begin{align}
	\frac{\partial\left(\mu_{0}  W\right)}{\partial\left(j_{\mathrm{NM}}\right)}=\xi_{\mathrm{DL}} \frac{\omega}{\gamma} \frac{\hbar}{2 e} \frac{\sin \theta}{\left(H_{\mathrm{FMR}}+M_{\mathrm{eff}} / 2\right) \mu_{0} M_{s} t_{\mathrm{FM}}}
\end{align}


\section{Multipul samples}

ST-FMR測定のSA比から分かる$\xi_\mathrm{FMR}$は
\begin{equation}
\xi_{\mathrm{FMR}}=\frac{S}{A}\left(\frac{e}{\hbar}\right) 4 \pi M_{s} t_{\mathrm{FM}}^{\mathrm{eff}} d_{\mathrm{NM}} \sqrt{1+\left(4 \pi M_{\mathrm{eff}} / H_{0}\right)}
\end{equation}
のようになり,このS,Aをあらわに書くと
\begin{equation}
S=\frac{\hbar}{2 e} \frac{\xi_{\mathrm{DL}} J_{e}^{\mathrm{rf}}}{4 \pi M_{s} t_{\mathrm{FM}}^{\mathrm{eff}}}
\end{equation}
\begin{eqnarray}
A &=&\nonumber \left(H_{T}+H_{\mathrm{Oe}}\right) \sqrt{1+\left(4 \pi M_{\mathrm{eff}} / H_{0}\right)} \\
&=&\left(\frac{\hbar}{2 e} \frac{\xi_{\mathrm{FL}} J_{e}^{\mathrm{rf}}}{4 \pi M_{s} t_{\mathrm{FM}}^{\mathrm{eff}}}+\frac{J_{e}^{\mathrm{rf}} d_{\mathrm{NM}}}{2}\right) \sqrt{1+\left(4 \pi M_{\mathrm{eff}} / H_{0}\right)}
\end{eqnarray}
となるので,ここから強磁性体の依存性は
\begin{equation}
\frac{1}{\xi_{\mathrm{FMR}}}=\frac{1}{\xi_{\mathrm{DL}}}\left(1+\frac{\hbar}{e} \frac{\xi_{\mathrm{FL}}}{4 \pi M_{s} t_{\mathrm{FM}}^{\mathrm{eff}} d_{\mathrm{NM}}}\right)
\end{equation}
となる~\cite{PhysRevB.92.064426}.
また,常磁性体の抵抗率$\rho_\mathrm{NM}$が分かっていれば
\begin{equation}
\frac{1}{\xi_{\mathrm{FMR}}}=\frac{1}{\xi_{\mathrm{DL}}^{E}}\left(\frac{1}{\rho_{\mathrm{NM}}}+\frac{\hbar}{e} \frac{\xi_{\mathrm{FL}}^{E}}{4 \pi M_{\mathrm{s}} t_{\mathrm{FM}} t_{\mathrm{NM}}}\right)
\end{equation}

と書き換えられる.$\xi_{\mathrm{DL,FL}}^{E}$は試料に印加された電場あたりのSOT変換効率と考えられ,単位がspin-Hall conductivityと同じになる.さらに,spin-Hall angleと$\xi_\mathrm{DL}$の関係が$G_{\mathrm{NM}} \equiv \sigma_{\mathrm{NM}} / \lambda_{\mathrm{s}, \mathrm{NM}}$であることを考えて,

\begin{eqnarray}
\xi_{\mathrm{DL}} &=\nonumber \theta_{\mathrm{SH}}\left(2 G^{\uparrow \downarrow} / G_{\mathrm{NM}}\right) /\left(1+2 G^{\uparrow \downarrow} / G_{\mathrm{NM}}\right) \\
&=\theta_{\mathrm{SH}}\left(2 G_{\mathrm{eff}}^{\uparrow \downarrow} / G_{\mathrm{NM}}\right) \equiv \theta_{\mathrm{SH}} T_{\mathrm{int}}
\end{eqnarray}
とかけることから,
\begin{eqnarray}
\xi_\mathrm{DL}^E = T_\mathrm{int}\sigma_\mathrm{SH}	
\end{eqnarray}
と表せる.この$T_\mathrm{int}$は,下で述べるようにspin-mixing conductanceが分かると定量できる.\\



さらにお得なことに,強磁性体膜厚依存性は磁化やダンピングの詳細な定量の可能性も与える.spin pumpingの理論からFMの隣にあるNMなどのspin sinkになり得る層が存在するときダンピングはFM単層の時より増大する.これは界面におけるspinの受け渡されやすさ(effective spin-mixing conductance $g_{\mathrm{eff}}^{\uparrow \downarrow}$)として表現できて,

\begin{equation}
g_{\mathrm{eff}}^{\uparrow \downarrow}=\frac{4 \pi M_{s} t_{\mathrm{FM}}^{\mathrm{eff}}}{\gamma \hbar}\left(\alpha-\alpha_{0}\right)=\frac{4 \pi M_{s} t_{\mathrm{FM}}^{\mathrm{eff}}}{\gamma \hbar} \Delta \alpha
\end{equation}
と定量できる.結局実際はST-FMR信号の線幅の周波数依存性から求めた$\alpha$の強磁性体膜厚依存性によって
\begin{equation}
\alpha = \frac{\gamma \hbar g_{\mathrm{eff}}^{\uparrow \downarrow}}{4 \pi M_{s} }\frac{1}{t_{\mathrm{FM}}^{\mathrm{eff}}}+ \alpha_0
\end{equation}
この傾きから計算できる.この$\alpha_0$はFMのintrinsicなダンピングである.
さらに,effective spin-mixing conductance $G_{\mathrm{eff}}^{\uparrow \downarrow}$は
\begin{equation}
G_{\mathrm{eff}}^{\uparrow \downarrow}= \frac{e^2}{h}g_{\mathrm{eff}}^{\uparrow \downarrow}
\end{equation}
となる.(実際はGの方が先に定義されて,ダンピングから出てくるgが後から定義される.)参考までに,これはPt/CoFeとかだと,$g$は$10^{19}$ m$^{-2}$弱,$G$は$10^{15}$ $\Omega^{-1}$m$^{-2}$弱くらいの大きさ.
さらに,NMの拡散長$\lambda_\mathrm{NM}$がわかれば,
\begin{eqnarray}
G^{\uparrow \downarrow}=\frac{\frac{\sigma_{\mathrm{P}_{\uparrow}}}{2 \lambda_{\mathrm{S}, \mathrm{NM}}}\left(\frac{e^{2}}{h}\right) g_{\mathrm{eff}}^{\uparrow \downarrow}}{\frac{\sigma_{\mathrm{NM}}}{2 \lambda_{\mathrm{s}, \mathrm{NM}}}-\left(\frac{e^{2}}{h}\right) g_{\mathrm{eff}}^{\uparrow \downarrow} \mathrm{coth}\left(\frac{d_{\mathrm{NM}}}{\lambda_{\mathrm{s}, \mathrm{NM}}}\right)}=\frac{G_{\mathrm{eff}}^{\uparrow \downarrow}}{1-2 G_{\mathrm{eff}}^{\uparrow \downarrow} / G_{\mathrm{NM}}}
\end{eqnarray}

という関係から,spin back flowの存在も考慮したbareなspin-mixing conductance $G^{\uparrow \downarrow}$を求めることができる.


また,この$G^{\uparrow \downarrow}$が分かると,spin transparency $T_\mathrm{int}$を
\begin{equation}
T_{\mathrm{int}}=\frac{G^{\uparrow \downarrow} \tanh \left(\frac{d_{\mathrm{NM}}}{2 \lambda_{NM}}\right)}{G^{\uparrow \downarrow} \mathrm{coth}\left(\frac{d_{\mathrm{NM}}}{\lambda_{\mathrm{NM}}}\right)+\frac{G_{\mathrm{NM}}}{2}}
\end{equation}
のように求めることができる~\cite{Zhang10.1038.nphys3304}.これはもちろん0以上1以下である.以上の中ではSMLを考慮していない.
SMLの簡単な考慮の仕方はeffective spin-mixing conductanceに取り込み
\begin{eqnarray}
g_{\mathrm{eff}}^{\uparrow \downarrow} &=G^{\uparrow \downarrow}\left[1-(1-\delta)^{2} \varepsilon\right]
\end{eqnarray}
と書き換える方法である~\cite{Taoeaat1670}.ただし
\begin{eqnarray}
\varepsilon &=G^{\uparrow \downarrow} /\left[G^{\uparrow \downarrow}+\frac{2}{3} k_{\mathrm{F}}^{2} \frac{l_{\mathrm{mf}}}{\lambda_{\mathrm{NM}}} \tanh \left(\frac{t_{\mathrm{N}}}{\lambda_{\mathrm{NM}}}\right)\right]
\end{eqnarray}

である.$\epsilon$がspin back flowを表現していて,$\delta$がSMLを表していて0がSMLがなく,1が完全に界面でlossするということを考えている.$k_{\mathrm{F}}$や$l_{\mathrm{mf}}$はHall測定で求める.それは次の章でまとめる.\\




また,飽和磁化と界面の磁気異方性も求められて,ST-FMR信号の共鳴磁場の周波数依存性からKittelの式によって有効的な飽和磁化$\mu_{0} M_{\mathrm{eff}}$を得られるが,それの強磁性体膜厚依存性により
\begin{equation}
\mu_{0} M_{\mathrm{eff}}=\mu_{0} M_{\mathrm{s}}-\frac{2 K_{\mathrm{s}}}{M_{\mathrm{s}} t_{\mathrm{FM}}^{\mathrm{eff}}}
\end{equation}
という関係を使える.$K_\mathrm{s}$は界面の垂直磁気異方性エネルギーでPt/Coだと$1$ mJ/m$^2$程度である.

\chapter{Hall effectによる物理量の定量}
膜厚$d$の試料に磁束密度$B$,電流$I$を流したときに測定したHall電圧$V_\mathrm{H}$は
\begin{eqnarray}
	V_\mathrm{H} = \frac{R_\mathrm{H}I}{d}B
\end{eqnarray}
とかける.この比例係数$R_\mathrm{H}$をHall定数と呼び,キャリアの電荷$q$(電子ならマイナス,正孔ならプラス)と電子密度$n$で
\begin{eqnarray}
	R_\mathrm{H} = \frac{1}{qn}
\end{eqnarray}
とかけるから,Hall効果を測れば電荷密度を見積もれる.
さらに伝導率$\sigma$が分かっていれば,緩和時間$\tau$は
\begin{eqnarray}
	\tau = \frac{\sigma m }{q^2 n}
\end{eqnarray}
と計算できる.ここで$m = 9.11\times10^{-31}$ kgである.
フェルミ面付近の電子が伝導を担っていると考えられる系だと,フェルミ速度$v_\mathrm{F}$を計算できる.これはフェルミ波数$k_\mathrm{F}$を用いて
\begin{eqnarray}
	v_\mathrm{F} = \frac{\hbar k_\mathrm{F}}{m}
\end{eqnarray}
となる.ただし,$k_\mathrm{F} = (3\pi^2 n )^{1/3}$と計算する.これらから平均自由行程$l_\mathrm{mf}$も定量できて,
\begin{eqnarray}
	l_\mathrm{mf} = v_\mathrm{F}\times2\tau
\end{eqnarray}
となる.
\chapter{高調波測定(Harmonic measurement)によるSOTの定量}
\bibliography{ref.bib}

\end{document}

PHYSICAL REVIEW B 92, 064426 (2015)

PHYSICAL REVIEW B 92, 214406 (2015)

Role of transparency of platinum–ferromagnet interfaces in determining the intrinsic magnitude of the spin Hall e￿ect

Self-consistent determination of spin Hall angle and spin diffusion length in Pt and Pd: The role of the interface spin loss